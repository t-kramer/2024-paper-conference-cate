%! Author = t.kramer
%! Date = 15/09/2024

\section{spatial Thermal Autonomy}
\label{sec:sta}


% Metric
Building on the initial concept proposed by \citet{levitt_thermal_2013} and a similar metric from the field of daylighting \citep{heschong_approved_2012}, we define spatial Thermal Autonomy (sTA) as the \textit{percentage of floor area during annual occupied time where a thermal zone meets or exceeds a given thermal comfort criterion through passive means only}. This metric captures both the temporal and spatial variability of dynamic indoor environments, offering deeper insights into comfort-related building performance.

Traditional metrics, particularly those involving PMV, often unintentionally promote excessive energy use by prescribing narrow temperature ranges that frequently necessitate active air conditioning systems. In contrast, sTA introduces a paradigm shift by prioritizing the evaluation and optimization of passive design strategies before resorting to active systems to mitigate uncomfortable conditions. This approach fundamentally shifts the focus towards enhancing a building’s passive performance, advancing energy autonomy, and fostering resilience against energy demands.

By emphasizing passive solutions, sTA not only reduces dependency on mechanical systems but also supports the broader adoption of personal comfort systems (PCS). A major challenge for PCS adoption has been their integration into existing building designs. sTA addresses this by highlighting spatial variability, providing a clear framework for designing and seamlessly incorporating PCS into building operations, ultimately reducing the need for energy-intensive conditioning.

% Implementation in Building Simulation
Since sTA is a spatially-resolved metric, its computation relies on spatial thermal simulation data. We suggest using ... based on ... . While this algorithm ..., it ... .

Furthermore, we have developed a custom Python library that allows to calculate sTA based on any time-step and spatially-resolved simulation results. Using the library, sTA can be computed and visualised for any grid of indoor environmental data and can be combined with existing short-term indices like the adaptive comfort or temperature ranges.


