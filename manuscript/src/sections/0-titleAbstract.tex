%! Author = t.kramer
%! Date = 10/06/2023

\begin{frontmatter}

\title{Designing for spatial Thermal Autonomy and active occupant adaption: Re-thinking the role of thermal comfort evaluation in early planning stages}


\author[inst1]{Tobias Kramer\corref{correspondingauthor}}
\affiliation[inst1]{organization={School of Architecture \& Built Environment, Queensland University of Technology},%Department and Organization
            %addressline={2 George St}, 
            city={Brisbane},
            %postcode={4000}, 
            %state={QLD},
            country={Australia}}


\author[inst2]{Stefano Schiavon}
\affiliation[inst2]{organization={Center for the Built Environment, University of California, Berkeley},%Department and Organization
            %addressline={Random St}, 
            city={Berkeley},
            %postcode={0000}, 
            state={CA},
            country={USA}}


\author[inst1]{Veronica Garcia-Hansen}


\author[inst3]{Vahid M. Nik}
\affiliation[inst3]{organization={Division of Building Physics, Department of Building and Environmental Technology, Lund University},
%             %addressline={Random St}, 
            city={Lund},
%             %postcode={0000}, 
%             %state={Skane},
            country={Sweden}}


%% corresponding author
\cortext[correspondingauthor]{Corresponding author}
%\ead{support@elsevier.com}


%%%%%%%%%%%%%%%%%%%%%%%%%%%%%%%%%%


\begin{abstract}

Thermal comfort modelling is part of the early building design process and significantly influences operational energy use and occupation satisfaction. Current methods and metrics inadequately reflect thermal comfort's spatial and personal diversity, leading to the "one-size-fits-all" approach when climatizing indoor spaces. We addressed these limitations by proposing a novel framework for early thermal comfort modelling composed of: spatial thermal simulation, personal comfort classification, a data-driven prediction algorithm, integration of Personal Comfort Systems (PCS), and a new metric termed spatial Thermal Autonomy (sTA). We incorporated these components into \textit{comfortSIM}, an open-source Python library and tested its practical implementation for a reference building in Brisbane, Australia. Here, we found that (a) introducing a spatial dimension, including the sTA metric, refines predictions of indoor environmental conditions and reduces local discomfort during operation; (b) considering personal diversity in data-driven models significantly improves thermal preference prediction ($>64\%$ accuracy) compared to the conventional PMV model (28\%); and (c) in a temperate climate like Brisbane, combining passive design with \gls{pcs} yields substantial energy savings for space conditioning ($>45\%$), enhances sTA, and improves thermal comfort for various types of persona. Our research contributes to improving the conventional thermal comfort modelling process and supports a paradigm shift toward dynamic indoor environments that foster individual control and adaptation.

\end{abstract}


%%Research highlights
% \begin{highlights}
%  \item Research highlight 1
% \item Research highlight 2
% \end{highlights}

\begin{keyword}
\texttt Thermal comfort \sep Building performance evaluation \sep Data science
\end{keyword}


\end{frontmatter}
