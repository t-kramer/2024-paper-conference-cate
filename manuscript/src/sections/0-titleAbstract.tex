%! Author = t.kramer
%! Date = 15/09/2024

\begin{frontmatter}

\title{Spatial Thermal Autonomy (sTA): A New Metric for Enhancing Building Design Towards Comfort, Heat Resilience and Energy Autonomy}


\author[inst1]{Tobias Kramer\corref{correspondingauthor}}
\affiliation[inst1]{organization={Center for the Built Environment, University of California},%Department and Organization
            %addressline={Random St}, 
            city={Berkeley},
            %postcode={0000}, 
            state={CA},
            country={USA}}


\author[inst1]{Stefano Schiavon}

\author[inst1]{Charlie Huizenga}

\author[inst2]{Veronica Garcia-Hansen}
\affiliation[inst2]{organization={Queensland University of Technology},
%Department and Organization
            %addressline={Random St}, 
            city={Brisbane},
            %postcode={0000}, 
            state={QLD},
            country={Australia}}


%% corresponding author
\cortext[correspondingauthor]{Corresponding author}
%\ead{support@elsevier.com}


%%%%%%%%%%%%%%%%%%%%%%%%%%%%%%%%%%


\begin{abstract}

Achieving thermal comfort in buildings while maintaining energy efficiency is a critical challenge in architecture and engineering design and operation. Traditional thermal comfort metrics used in the early stages of design tend to neglect two key aspects: spatial variability of thermal conditions within buildings and the promotion of passive design strategies over active conditioning systems. This oversight leads to localized discomfort, excessive energy use, and increased vulnerability to overheating.
To address these issues, we propose a novel metric called spatial Thermal Autonomy (sTA). The sTA metric has two advantages over existing metrics. Firstly, it captures spatial variability in thermal conditions. Secondly, it quantifies how much a building is able to provide thermally comfortable conditions without the use of active sources of energy.
We performed a simulation case study evaluating sTA for different thermal zone sizes, passive design levels, and climate scenarios. Our findings suggest that buildings with high spatial thermal autonomy tend to use less energy, demonstrate greater thermal resilience during extreme weather or power outages, and experience fewer local discomfort problems.
Optimizing building designs for spatial Thermal Autonomy encourages passive design strategies in key decisions related to building form, envelope, conditioning strategies, and HVAC system design. In buildings with reduced heating and cooling loads, this approach supports the increased adoption of local low-energy personal comfort solutions, such as fans or local heating solutions, and can lead to more adaptive, resilient, and comfortable indoor environments in a changing climate.


\end{abstract}


%%Research highlights
% \begin{highlights}
%  \item Research highlight 1
% \item Research highlight 2
% \end{highlights}

\begin{keyword}
\texttt Thermal comfort \sep Building performance simulation \sep Thermal autonomy
\end{keyword}


\end{frontmatter}
