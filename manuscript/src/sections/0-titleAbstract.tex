%! Author = t.kramer
%! Date = 15/09/2024

\begin{frontmatter}

\title{Spatial Thermal Autonomy (sTA): A New Metric for Enhancing Building Design Towards Comfort, Heat Resilience and Energy Autonomy}


\author[inst1]{Tobias Kramer\corref{correspondingauthor}}
\affiliation[inst1]{organization={Center for the Built Environment, University of California, Berkeley},%Department and Organization
            %addressline={Random St}, 
            city={Berkeley},
            %postcode={0000}, 
            state={CA},
            country={USA}}


\author[inst1]{Stefano Schiavon}

\author[inst1]{Edward Arens}


%% corresponding author
\cortext[correspondingauthor]{Corresponding author}
%\ead{support@elsevier.com}


%%%%%%%%%%%%%%%%%%%%%%%%%%%%%%%%%%


\begin{abstract}

Achieving thermal comfort in buildings while maintaining energy efficiency is a critical challenge in modern architecture and engineering. Traditional thermal comfort metrics used in the early design stages often neglect two key aspects: the spatial variability of thermal conditions within buildings and the promotion of passive design strategies over active conditioning systems. This oversight leads to localized discomfort, excessive energy use, and increased vulnerability to climate change and extreme weather events in many modern buildings.
To address these issues, we propose a novel metric called spatial Thermal Autonomy (sTA). The sTA metric provides two primary advantages over existing metrics. Firstly, it captures spatial variability in thermal conditions, which is essential for optimizing comfort throughout an entire building and for a diverse group of occupants. Secondly, it emphasizes energy autonomy by reducing reliance on external energy sources while maximizing thermal comfort.
We conducted a simulation case study comparing sTA with standard long-term thermal comfort metrics across various building typologies and climates. Our findings show that buildings with a high spatial Thermal Autonomy tend to use less energy, demonstrate greater thermal resilience during extreme weather or power outages, and experience fewer issues with local discomfort.
Optimizing building designs for spatial Thermal Autonomy encourages passive design strategies in key decisions related to building form, envelope, conditioning strategies, and HVAC system design. This approach supports the increased adoption of local, low-energy personal comfort solutions, such as fans or local heating solutions, in buildings with reduced heating and cooling loads, and can lead to more adaptive, resilient, and comfortable indoor environments in a changing climate.


\end{abstract}


%%Research highlights
% \begin{highlights}
%  \item Research highlight 1
% \item Research highlight 2
% \end{highlights}

\begin{keyword}
\texttt Thermal comfort \sep Building performance evaluation \sep Data science
\end{keyword}


\end{frontmatter}
