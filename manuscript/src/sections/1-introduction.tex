%! Author = t.kramer
%! Date = 10/06/2023

\section{Introduction}

% CONTEXT
Indoor thermal environments are dynamic and tend to be spatially and temporally heterogeneous \citep{Mishra2016, Clements2019, Kim2019}. Further, the perception of these dynamic conditions is highly diverse and varies significantly between subjects \citep{Schweiker2018, Gauthier2020, Mishra2013}. In stark contrast, conventional thermal comfort evaluation methods during building design are characterised by over-generalisation and a "one-size-fits-all" approach. Designers often ignore the spatial thermal heterogeneity of dynamic indoor environments, apply aggregate thermal comfort models that inadequately reflect individual diversity and favour tightly controlled buildings instead of proactively fostering individual adaptation and passive design solutions. The consequence are monotonous buildings \citep{Altomonte2020} that leave 40\% of its occupants dissatisfied \citep{Graham2021} and actively promote excessive energy use and over-conditioning \citep{Parkinson2021, Sekhar2016, DerribleReeder2015} instead of providing occupant-centred, efficient and comfortable indoor environments.

% OPPORTUNITY
However, recent advances in low-cost sensor technology \citep{Ji2023}, data-driven approaches driven by data science \citep{Abdelrahman2021, LuoNa2021} and \gls{ai} \citep{Ngarambe2020} as well as the observable paradigm shift toward personalised thermal comfort solutions \citep{KimSchiavon2018} provide a unique opportunity to fundamentally rethink current conventions on early thermal comfort design, evaluation, and planning. 

% AIM & OBJECTIVES
In this work, we propose a framework for a more occupant-focused prediction and evaluation of thermal comfort during building design. To this end, our objectives were (1) to describe the problems of conventional thermal comfort evaluation in the building design phase; (2) to develop a novel workflow of early thermal comfort evaluation that promotes the combination of passive design solutions and active occupant adaptation; and (3) to test the practical implementation of the proposed method.


% OUTLINE
% Initially, we outline the contextual problem of our study and contrast the issues of common thermal comfort prediction methods with the opportunities provided by recent technological trends. We then introduce our proposal for a spatially resolved and personalised approach to early thermal comfort evaluation. This is followed by a test of the methods practical implementation in a case study. Lastly, we discuss emerging questions around the adoption of our framework into practice.
