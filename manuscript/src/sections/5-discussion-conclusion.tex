%! Author = t.kramer
%! Date = 15/09/2024


% \pagebreak
\section{Conclusion}

% Summary of the findings
\subsection{Summary of the findings}

The spatial Thermal Autonomy (sTA) metric proposed in this paper offers several advantages that make it a valuable tool for optimizing thermal comfort and energy efficiency in building design. First, it leads to a more precise representation of indoor thermal comfort by capturing the spatial variability of thermal conditions. Unlike conventional metrics and workflows, which provide a single point temporal assessment of comfort over time, sTA offers a more nuanced and comprehensive view by accounting for spatial thermal differences within zones. Our findings reveal that larger zones experience more thermal heterogeneity, while buildings with higher construction standards consistently show reduced temperature variations and higher sTA values. This suggests that improving construction quality can mitigate local discomfort by creating awareness of existing thermal heterogeneity.

Moreover, using adaptive or field-data-driven comfort models resulted in higher sTA values and responsiveness to temperature changes compared to traditional PMV-based metrics. Based on our findings, the PMV lacks sensitivity to account for spatial differences. This could be caused by the significant number of PMV input parameters that are not directly affected by spatial temperature differences, especially clothing and metabolic rate, which are often assumed and treated as constants for long-term thermal comfort assessment. Moreover, PMV has been shown to have low prediction accuracy \citep{cheung_analysis_2019}. This is why we recommend using the adaptive model as the comfort criterion for sTA.

In addition, sTA encourages the use of passive design solutions and establishes a clear link between passive design quality, long-term thermal resilience, and energy performance. Our analysis showed that improved construction standards and higher sTA significantly reduced cooling energy demand, with energy savings of up to 46\% under current climate conditions and 37\% in future climate scenarios. This highlights the potential of sTA to support energy-efficient building design that adapts to both present and future climate challenges. 

Furthermore, our results show that optimizing building design towards higher sTA can mitigate the effects of rising temperatures, both over time and across zones. The analysis of passive operative temperature distributions highlighted that the design guided by sTA contributes to maintaining comfortable indoor environments, even under more extreme future weather conditions. In practice, this might improve building resilience and shows that occupants can remain comfortable with minimal reliance on active conditioning systems in passively well-designed buildings.


% from earlier section
% By emphasizing passive solutions, sTA not only reduces dependency on mechanical systems but also supports the broader adoption of personal comfort systems (PCS). A major challenge for PCS adoption has been their integration into existing building designs. sTA addresses this by highlighting spatial variability, providing a clear framework for designing and seamlessly incorporating PCS into building operations, ultimately reducing the need for energy-intensive conditioning.




% Implications for Building Design
\subsection{Implications for building design}


The spatial Thermal Autonomy (sTA) metric offers a novel approach to improving thermal comfort and energy efficiency in building design. By capturing spatial variability, sTA enables a more accurate representation of indoor thermal conditions compared to conventional, PMV-based metrics, which only assess single-point comfort over time. This enhanced representation allows for better identification and management of thermal heterogeneity, particularly in larger zones or areas with high exposure, such as near windows, and can help mitigate potential local discomfort. Buildings with higher sTA values consistently demonstrated reduced temperature variability, particularly when construction standards were improved, which suggests that enhancing building envelopes can effectively address these comfort issues.

In addition to optimizing comfort, sTA encourages passive design strategies that reduce energy demand. Prioritizing elements such as improved insulation, shading, and optimal window placement can lead to significant reductions in cooling energy use. Our findings show that buildings optimized for higher sTA values experienced energy savings of up to 46\% in current climates and 37\% under future climate scenarios. This strong correlation between sTA and energy performance underscores its utility in guiding long-term energy-efficient design.

sTA also offers a forward-looking tool to address future climate challenges. By simulating spatial thermal heterogeneity, sTA helps architects and engineers assess how a building will perform under extreme weather conditions, supporting decisions that enhance resilience. Buildings designed with higher sTA values showed better performance in maintaining comfort during future climate scenarios, demonstrating the potential to reduce reliance on active conditioning systems and improve overall thermal resilience.

The emphasis of the metric on passive design strategies and spatial variations aligns well with the growing use of personal comfort systems (PCS), such as fans and localized heating solutions. sTA highlights areas where personal comfort systems can be effectively integrated into building operations, helping to minimize energy use while maintaining occupant comfort in spaces with varying thermal conditions.

In general, our findings suggest that by incorporating sTA early in the design process, building projects can effectively balance thermal comfort, energy efficiency, and resilience. This holistic approach to design addresses current and future climate challenges and promotes indoor environments that are adaptive, sustainable, and comfortable over time and space.





% Limitations and future work
\subsection{Limitations \& future work}

While this study demonstrates the potential of spatial Thermal Autonomy (sTA) in optimizing thermal comfort and energy efficiency, there are several limitations that need to be addressed in future research.

Firstly, the analysis focused on a specific building typology and climate zone — namely, Sydney, Australia. To broaden the applicability of sTA, future studies should investigate different building types and climates to ensure that the metric is robust and adaptable in various contexts.

Secondly, although sTA provides a spatially resolved comfort metric, it relies heavily on thermal simulations, which may not fully capture the complexities of dynamic occupant behavior. Incorporating better models of adaptive behaviors into future studies would provide a more comprehensive understanding of comfort and further validate the use of sTA in diverse building environments.

Finally, this study separately assessed passive and active conditioning systems. Investigating hybrid systems and personal comfort systems in relation to sTA could provide valuable insights into how these approaches interact to enhance thermal autonomy, comfort, and energy efficiency, especially in future climate scenarios.